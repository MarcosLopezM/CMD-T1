\documentclass[./../main.tex]{subfiles}
\graphicspath{{img/}}

\begin{document}
    \section{}

    La delta de Dirac \(\delta(x - x_{0})\) tienen la propiedad de que \(\int_{-\infty}^{\infty}\delta(x - x_{0})f(x)\odif{x} = f(x_{0})\), para una función bien comportada.

    \begin{enumerate}
        \item Prueben que
        
        \begin{equation*}
            \int_{-\infty}^{\infty}\delta(f(x))\odif{x} = \dfrac{\delta(x - x_{0})}{\abs{f^{\prime}(x_{0})}}.
        \end{equation*}

        Para una función derivable que admite inversa y en donde \(x_{0}\) es una raíz: \(f(x_{0}) = 0\). \emph{Sugerencia}. Hagan el cambio de variable \(u = f(x)\). A notar que el valor absoluto aparece porque la delta de Dirac es siempre positiva.

        \item Prueben ahora que 
        
        \begin{equation*}
            \int\odif{p^{0}}\odif[order=3]{p}\, \delta((p^{0})^{2} - {\vect{p}}^{2} - m^{2}) = \int\odif[order=3]{p}\dfrac{1}{2E},
        \end{equation*}

        en donde \(E = \sqrt{{\vect{p}}^{2} + m^{2}}\).
    \end{enumerate}
\end{document}