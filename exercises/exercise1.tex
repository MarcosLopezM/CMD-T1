\documentclass[./../main.tex]{subfiles}
\graphicspath{{img/}}

\begin{document}
    \section{}

    Considerar el campo de desplazamientos definido por:

    \begin{equation*}
        \vect{u}(x, y, t) = u(x, y, t)\ex + v(x, y, t)\ey
    \end{equation*}

    donde

    \begin{alignat*}{2}
        u(x, y, t) &= \sin\bigl(\tfrac{2\pi x}{L}\bigr)\cos\bigl(\tfrac{2\pi y}{L}\bigr) &\qquad\qquad & v = -\cos\bigl(\tfrac{2\pi x}{L}\bigr)\sin\bigl(\tfrac{2\pi y}{L}\bigr) \\
    \end{alignat*}

    en el dominio \(0 \leq x \leq L_{x} = 1\), \(0 \leq y \leq L_{y} = 1\).
    
    Determinar:

    \begin{enumerate}[label=\arabic*)]
        \item el tensor de gradiente del campo de desplazamientos,
        \item el tensor de deformación (parte simétrica del gradiente del campo de desplazamientos),
        \item el tensor de rotación (parte antisimétrica del gradiente del campo de desplazamientos),
        \item el vector de rotación,
        \item la divergencia. 
    \end{enumerate}

    Graficar:

    \begin{enumerate}[resume, label=\arabic*)]
        \item campo de desplazamientos,
        \item vector de rotación (nota: considerar como un campo escalar).
    \end{enumerate}
\end{document}